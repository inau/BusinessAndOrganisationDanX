\begin{thebibliography}{9}

\bibitem{gert001}
	Malene(0 - 21:03), Gert(21:03 - 1:13:20) \& Lasse(1:13:20 - slut).3ga - Time 24:47\\
	\textit{vi henter jo hver dag nede i Holland og leverer i hele Danmark inden 7.}

\bibitem{gert002}
	Malene(0 - 21:03), Gert(21:03 - 1:13:20) \& Lasse(1:13:20 - slut).3ga - Time 43:00\\
\textit{hente nede i Salzgitter som ligger nede syd for Hannover klokken 15:30 og så samtidig levere oppe nord for Stockholm inden 7 næste morgen.}

\bibitem{gert003}
	Gert \#2 2-2.3ga - Time 02:17\\
	\textit{[...], så skal man altså også have løsningsdelen med, hvornår det er løst og sådan, før det har en værdi der. Og hvis man ikke gør det, så føler jeg at det mister meget af sin værdi.}

\bibitem{gert004}
Gert \#2 2-1.3ga - Time 03:10\\
	\textit{[...] og hvis de ringer til vores hovednummer er det control tower de får, så det er jo egentlig der den meste kontakt foregår}

\bibitem{gert005}
	Gert \#2 2-1.3ga - Time 04:15\\
	Om kontroltårnet selv løser problemer for kunder: \textit{Det kunne sagtens være, der hvor kunden siger noget, kan vi godt sige til dem, jamen prøv at hør, hvis du nu går ind på vores hjemmeside, så kan du gøre sådan og sådan}

\bibitem{gert006}
Gert \#2 2-1.3ga - Time 06:21\\
	\textit{Hvis jeg får noget fra en intern eller en kunde prøver jeg på en eller anden måde at oversætte det sådan at det er mere klart defineret når det kommer til IT}

\bibitem{gert007}
	Gert \#2 2-1.3ga - Time 03:30\\
	\textit{(om kunden) som så skyder det ind på et andet niveau her, for eksempel ved at kontakte mig eller Bob}

\bibitem{gert008}
	Gert \#2 2-1.3ga - Time 04:10\\
	Om kontroltårnet kontakter operationen: 
	\textit{Ja, kontrol tårnet er vores customer service og operation og alt det der.}

\bibitem{gert009}
	Gert \#2 2-1.3ga - Time 14:45\\
	Om de får den samme support request flere gange: \textit{Hvis du får den samme forespørgsel flere gange så er det jo en rykker, enten fordi vi snakkede om det dengang eller fordi vi ikke lige har fået set på det, og det så bliver nævnt igen. Også siger vi jo det skal vi nok prøve at se på eller komme tilbage til også er vi ikke lige kommet tilbage.} Interviewer: Hvad er årsagen til det? \textit{Det er fordi hvis jeg sender en mail til Lasse så er det ikke sikkert jeg husker at følge op på det, og hvis han ikke svarer, så husker jeg det først når kunden rykker på mig, så vi mangler et struktureringsværktøj. } 

\bibitem{bob001}
	\textit{Nogle potentielle kunder vil gerne se support KPIer.}

\bibitem{bob002}
	\textit{Nogle kunder kræver referencer fra andre kunder.}

\bibitem{bob003}
	\textit{Dvs at vi fokusere på at vokse med ca. 25-30 om året i organisk vækst (nye kunder)}\\
\textit{Sidst eår tjene vi ca 10 mill indne skal, hvilket er for lidt ud af 250 mill i omsætning. I indeværende år (13/14) kommer vil til at omsætte plus 300 mill og et overskud på plus 20 mill før skat.}\\
\textit{Overordnede sigter jeg mod at DANX om 3-4 år har en omsætning på plus 500 mill med et overskud på ca 8-10 procent. Det vil give en potentiel salgsværdig på + 500 mill.}

\bibitem{malene001}
	Malene(0 - 21:03), Gert(21:03 - 1:13:20) \& Lasse(1:13:20 - slut).3ga - Time 09:23

\bibitem{lahib001}
	Lahib.3ga - Time 13:00

\bibitem{lasse001}
	Lasse5.3ga - Time 15:25 \\
	Interviewer: \textit{Hvor ofte er operationen mellemled for support henvendelser fra kunden?} Lasse: \textit{Stort set altid, vi snakker 90\% af gangene i hvert fald}

\bibitem{lasse002}
	Lasse5.3ga - Time 8:27 \\
	Til om nogen fra operationen kan løse it-problemer: \textit{Det er igen lidt et tilfælde hvem der kan det og ikke kan det.}

\bibitem{lasse003}
	Lasse5.3ga - Time 16:08 \\
	\textit{Jeg har jo lidt et problem eller en udfordring med ved at vi har to afdelinger både en IT support og en IT development, og de sender jo bare til @it som har begge afdelinger blandet sammen så vi for jo en masse mails hvis der er for eksempel problemer med PDA’erne eller et andet hvor det sådan set ikke er min afdeling, så vi får en masse clutter-mail på grund af det her.}

\bibitem{lasse004}
	Lasse5.3ga - Time 17:10 \\
	\textit{Hvis de andre har travlt så svarer vi jo også, vi hjælper jo hinanden (IT afdelingerne)}

\bibitem{lasse005}
	Lasse5.3ga - Time 18:30 \\
	Interviewer: \textit{Hvor de(operationen) har korrespenderet med en kunde?} Lasse: \textit{Ja, men hvor de så også skriver hvad det er der er problemet.}

\bibitem{lasse006}
	Lasse5.3ga - Time 21:35 \\
	\textit{Jeg har jo lært for eksempel haydar(ansat i operationen) at gå ind at kigge i EDI filen hvis der er fejl advisering eller et eller andet så han kan gå ind og se for vi gemmer dem et bestemt sted og han er begyndt at kunne gennemskue de fleste af dem, altså se hvordan ser denne her forsendelse egentlig ud, så han har taget meget af det jeg har siddet og gennemtrawlet og se om de har sendt den her forsendelse gennem de rigtige kanaler, er den i vores fil og i så fald hvad er der så gået galt.}

\bibitem{lasse007}
	Malene(0 - 21:03), Gert(21:03 - 1:13:20) \& Lasse(1:13:20 - slut).3ga - Time 01:50:38
	\textit{Nogle gange kunne det være meget rart for mig at sige, prøv at se her hvor meget vi egentlig har lavet, og her kunne et ticket system jo være meget relevant.}

\bibitem{lasse008}
	Lasse5.3ga - Time 03:13 \\
	\textit{Hvis ikke IT integrationen er færdig, jamen så kører vi ud med pakkerne alligevel, altså så gør vi det uden. Så vi kan som regel integrere en ny kunde, hvis de er klar, inden for en til to dage.}

\bibitem{webpage001}
	http://www.tnt.com/express/da\_dk/site/home/vores\_services/TNT\_innight.html
\bibitem{webpage002}
	http://www.postnordlogistics.dk/da/Sider/hit.aspx
\bibitem{webpage003}
	http://www.posten.se/en/Logistics/InNight/Pages/home.aspx
\bibitem{webpage004}
	http://www.fedex.com/dk/shipping-services/domestic/
\bibitem{webpage005}
	http://www.kayako.com/
\bibitem{webpage006}
	http://www.kayako.com/company/customers/
\bibitem{webpage007}
	http://www.g2crowd.com/survey\_responses/kayako-fusion-review-10500
\bibitem{webpage008}
	http://www.kayako.com/signup/download/case/

\bibitem{mail}
\textit{Vi har generelt en vækst strategi. Se evt power point vedhæftet. Dvs at vi fokusere på at vokse med ca. 25-30 om året i organisk vækst (nye kunder). I indeværende regnskabsår sætter vi også fokus på vores indtjening. Sidste år tjente vi ca 10 mill inden skat, hvilket er for lidt ud af 250 mill i omsætning. I indeværende år (13/14) kommer vil til at omsætte plus 300 mill og et overskud på plus 20 mill før skat.\\
Overordnede sigter jeg mod at DANX om 3-4 år har en omsætning på plus 500 mill med et overskud på ca 8-10 procent. Det vil give en potentiel salgsværdig på + 500 mill.\\
Det er vores egentlig målsætninger. De andre ting er mere værdier for os. Vi arbejder med nedenstående values i virksomheden:
\begin{description}
\item[Reliability]
We have the integrity to keep our promises, to correct our mistakes and proactively inform our customers.
\item[Equality]
We are all equals, performing different roles to achieve the same goal.
We treat our customer, partners and colleagues with the same respect that we want to achieve ourselves.
\item[Quality]
We never take our customers for granted.
We strive for 100\% in everything we do – in that way we ensure that our customers live up to their customers’ high expectations.
\item[Flexibility]
Flexibility is our mindset. It is what we are and what we expect from all our employees and partners - nothing less.
\item[Creativity]
As pioneers we think out of the box and create solutions for our customers’ needs.
We are never satisfied and are constantly looking for new ways to improve.
\item[Availability]
We ensure our customers’ availability of spare parts through each employee’s personal care and availability.
\item[Pride]
We are proud of our customers, our company and our people - we take pride in everything we do.
\end{description}
Bob Thorhauge}
	


\end{thebibliography}
