\chapter{Project charter}

\section{Introduction}
The project is being undertaken for learning purposes in order to simulate the actual work done in the MUST method for the course Business Processes and Organisation, Fall 2013, at the IT university of Copenhagen. The project will not produce any final products other than a business case and associated appendix. The project will be supervised by Nina Boulus-Rødje and Elisabeth Broe Christensen.

\section{Purpose of project charter}
The project charter defines the objective of the project and what work has to be done to reach this objective, as well as which resources the project team needs from DANX and which deliverables the company will receive. The project charter serves as a contract between the project team and DANX, such that everyone involved in the project has the same view of the vision, assumptions, planning, resource consumption and boundaries of the project.

\section{Premise}
Background

\subsection{Assignment and Objective}
The employees of DANX can not access error reports from a specific customer, because some of the reports are not documented and the current system does not allow reports to be retrieved within a satisfactory time frame. This hinders potential evaluation of support given to a customer.\\
How can employees of DANX access documentation for customer support?

\subsection{Critical success factors}
The problem is solved if the following goals are reached.
\begin{itemize}
	\item It is possible for at least one employee to retrieve documentation for customer support.
	\item The customer support documentation provides data to evaluate how fast it is done, who it is done by, and who receives it.
	\item The solution does not decrease the service level of the customer support.
\end{itemize}

\subsection{Scope}
The scope of the project includes internal communication, not communication with customers and others outside of DANX. In other words this means that customers will still have the same high degree of service and availability from all IT-staff, because the proposed solutions do not seek to change the way customers communicate with the employees of DANX. The regional departments outside denmark will not be considered for the problem or the solution.

\subsection{Methods}
A solution to the problem is proposed based on research including interviews of certain employees and observations of activities relevant to the problem.

\subsection{Background}
DANX is a Nordic logistics company. They specialize in express delivery the following day within 7 AM. DANX operates in Denmark, Sweden, Norway, and Finland, and is managed by DANX Nordic.

\subsection{SWOT}
Strengths and weaknesses are internal in the company, where opportunities and threats are external factors.

\begin{tabular}{| p{\dimexpr.5\textwidth-2\tabcolsep} | p{\dimexpr.5\textwidth-2\tabcolsep} |}
\hline
\rowcolor{GR}
\textbf{Strengths} & \textbf{Weaknesses} \\ \hline
Lots of failover backup plans, resulting in 99.5\% on-time delivery & Lots of internal programs.
\\ \hline
Lots of PUDOs and FLS. & The IT-developent department has to spend time on IT-support tasks \\ \hline \hline
\rowcolor{GR}
\textbf{Opportunities} & \textbf{Threats} \\ \hline
Go to other ELSA companies, and obtain know-how & Restricted access to customer IT-system APIs \\ \hline
\end{tabular}
\qquad

As seen on the SWOT table the IT-development department has to spend time on IT-support tasks. This time could have been spent on developing integration for a client, which eventually could leave to more income.