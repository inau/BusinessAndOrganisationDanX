\chapter{Project charter}
The project is being undertaken for learning purposes in order to simulate the actual work done in the MUST method for the course Business Processes and Organisation, Fall 2013, at the IT university of Copenhagen. The project will not produce any final products other than a business case and associated appendix. The project will be supervised by Nina Boulus-Roedje and Elisabeth Broe Christensen.

\section{Purpose of project charter}
The project charter defines the objective of the project and what work has to be done to reach this objective, as well as which resources the project team needs from DANX and which deliverables the company will receive.

The project charter serves as a contract between the project team and DANX, such that everyone involved in the project has the same view of the vision, assumptions, planning, resource consumption and boundaries of the project.


\section{Premise}
\subsubsection{Background}
DANX is a Nordic transportation company. They specialize in express delivery the following day within 7 AM. DANX operates in Denmark, Sweden, Norway, and Finland, and is managed by DANX Nordic.
 
\subsubsection{Scope}
The project group decided to scope the project in such a manner that the proposed area for improvement only applies to internal communication to the IT-department and not external. In other words this means that customers will still have the same high degree of service and availability from all IT-staff whereas the internal communication should be handled in a fashion where the developer stress is reduced or avoided. 

	Technical factors

\subsection{Critical factors}
\subsubsection{Critical success factors}
\begin{itemize}
	\item The time spent on development must be increased by 10% after five years.
	\end{itemize}

\subsubsection{Critical preconditions}
\begin{itemize}
	\item It must be possible for the project group to observe how the flow of DANX is on a regular day.
	\item The project group can conduct interviews with relevant persons in order to obtain knowledge about DANX.
	\end{itemize}

\subsection{Assignment and Objective}
The IT-development department of DANX serves many purposes, some of which are crucial for delivery of the company’s service.

However the department handles problems which are less relevant for the delivery of the service, which has potential to be optimized* or removed from the responsibilities of the IT department. \\

Employees of the IT development department spend time on prioritizing, investigating and solving problems that are not related to IT development and customer service, which can be reduced. \\

How do the employees of the IT development department reduce the time spent on solving problems that are not related to IT development or customer service?
\\ \\
The problem is solved if the IT-department uses 15\% less time on solving problems not related to IT-development or customer service one year after the solution has been deployed.
\\
\small{*time spent of solving problem reduced.}\\

\subsubsection{Methods}
A solution to the problem is proposed based on research including interviews of certain employees and observations of activities relevant to the problem.

\definecolor{GR}{gray}{0.85}

\subsubsection{SWOT}
Strengths, Weaknesses, Opportunities, and Threats.
Strengths and weaknesses are internal in the company, where opportunities and threats are external factors.

\begin{table}[htdp]
\label{SWOT analysis}
\begin{tabular}{| p{\dimexpr.5\textwidth} | p{\dimexpr.5\textwidth-4\tabcolsep} |}
\hline
\rowcolor{GR}
\textbf{Strengths} & \textbf{Weaknesses} \\ \hline \hline
Lots of failover backup plans, resulting in 99.5\% on-time delivery & Lots of internal programs.
\\ \hline
Lots of PUDOs and FLS. & The IT-developent department has to spend time on IT-support tasks \\ \hline \hline
\rowcolor{GR}
\textbf{Opportunities} & \textbf{Threats} \\ \hline
Go to other ELSA companies, and obtain know-how & Restricted access to customer IT-system APIs \\ \hline
\end{tabular}
\end{table}

As seen on the SWOT table the IT-development department has to spend time on IT-support tasks. This time could have been spent on developing integration for a client, which eventually could leave to more income.

\subsection{Agreements and coordination}
The project groups contact person in DANX is marketing manager Malene Vig Hjarnaa. she will be responsible for providing contact to other employees inside the company and making the required resources available. Malene can be contacted on +45 …, during regular office hours, i.e. 0900-1600.

\subsection{Stakeholders}
The stakeholders of this design project is the IT department of DANX. They are the ones who will be immediately affected by the project.

\subsection{Intermediate products}
\begin{description}
	\item[Project charter] 
	\item[Strategic alignment report] 
	\end{description}